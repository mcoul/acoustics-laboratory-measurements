	A partir de las mediciones realizadas, se llegó a caracterizar los paneles en las siguientes categorías:
	
		\begin{table}[h!]
			\centering
			\begin{tabular}{ccc}
			\toprule
			\textbf{Característica} & \textbf{Requerimiento} & \textbf{Especificación alcanzada}\\
			\midrule
			Aislamiento al ruido aéreo & Categoría B3 & Categoría B3\\
			Absorción sonora & Categoría A3 & Categoría A2\\
			\bottomrule
			\end{tabular}
		\end{table}
		
	Como ya se concluyó en las correspondientes secciones, los paneles ensayados cumplen sólo con uno de los dos requerimientos, por lo que no pueden considerarse aptos para ser usados en la obra.\\
	
	Por otro lado, si se la instalara en una autopista, y los requerimientos acústicos fuesen los mismos que para la obra ferroviaria, habría que tener en cuenta el \textsc{espectro normalizado de ruido de tránsito}, definido en la norma \texttt{UNE-EN 1793-3}, en lugar del \textsc{espectro normalizado de ruido ferroviario}, definido en la norma \texttt{UNE-EN 16272-3}. Esto se realiza para respetar la norma \texttt{UNE-EN 1793-2}, en donde se establece que el espectro de ruido normalizado a utilizar en el cálculo de los parámetros globales depende del uso al que estén destinados. Como se mostró para los cálculos de los índices, se ve que utilizando el espectro de ruido de tráfico, los paneles corresponden a la misma categoría, a pesar de haber una leve variación en el valor del índice calculado. También se puede decir que el ruido de tráfico rodado tiene más componentes graves que el de los ferrocarriles, porque en éste último hay una rueda de metal con vías de metal.\\

	Para el caso de la absorción sonora, la muestra cambia de categoría; por esta razón, se podría concluir en que la barrera estudiada no tiene el mismo comportamiento frente a los dos tipos de ruido (de tráfico y ferroviario).